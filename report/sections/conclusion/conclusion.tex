\section{Conclusion}\label{sec:conclusion}

Representing \acp{kg} as low-dimensional vector embeddings has its computational advantages.
Querying existing knowledge is fast since arithmetic operations on embedded entities and relations act as translation from head entity to tail entity and vice versa.
The \ac{hake} technique, furthermore, has the advantages of being able to model hierarchical knowledge and complex relations.
However, the technique is not very precise and without a successful replication experiment, we doubt that the technique is flexible and scalable enough to be implemented in real-world use cases.
This argument is substantiated by the fact that performance on domain-specific data is significantly worse than on general-domain knowledge found in standard datasets.
We conclude that embedding technique might not be a suitable method for the link prediction problem in \acp{kg}.
Therefore, a deeper investigation into correct methods for the problem is necessary.
Any future solution to this problem should consider not only the structure of the graph, but also the semantics of entities and their relations, for it to work effectively.
