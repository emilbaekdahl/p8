\section{Introduction}\label{sec:introduction}

Representation of knowledge in a way that suits computation is at the core of many machine intelligence tasks and agents.
A collection of such knowledge is called a \ac{kb} and it describes how different entities in a domain are related.
A \ac{kb} can be constructed either manually or automatically by computer programs that extract knowledge from existing sources.
Either way, during the construction, there might be a trade-off between completeness and correctness\cite{Paulheim2016}.
Completeness describes how well a \ac{kb} captures all knowledge in the desired domain while correctness describes the degree of truth of the knowledge.
When creating a \ac{kb} by hand, it is possible to manually verify the correctness of the knowledge.
However, it can be an infeasible task to manually cover all knowledge in larger domains, resulting in an incomplete \ac{kb}.
To speed up this process, automatic methods, such as building computer programs that extract knowledge, can be used.
However, while automatically extracting knowledge from many different sources can give a complete \ac{kb}, it may result in creating false knowledge.
The trade-off lies in finding a reasonable balance between correct and complete knowledge.

As a consequence of this issue, it is attractive to develop solutions that improve correctness or completeness of \acp{kb}.
This task is called \ac{kb} refinement and approaches to it are either concerned with
\begin{enumerate*}
  \item increasing completeness by inferring new or hidden knowledge from existing knowledge, or
  \item increasing correctness by finding and removing false knowledge.
\end{enumerate*}
Especially the former sees a gain in popularity since it can provide otherwise undiscovered insight into existing data and thus help humans in decision making.

One approach to computationally increasing completeness is through link prediction.
The goal of this task is to infer new knowledge by identifying missing links between existing entities in a \ac{kb}.
For link prediction, the current most popular methods are based on vector embeddings of \acp{kb}.
In this project, we take a closer look at this state-of-the-art technique.
Specifically, we focus on a recent technique that uses a \ac{hake} of \acp{kb} to infer missing links between entities in a hierarchical graph structure\cite{Zhang2019}.
We explain the underlying concepts for this technique and conduct experiments with existing datasets and a new domain-specific dataset.

The rest of this report is structured as follows.
Section~\ref{sec:related-work} discusses existing work on \acp{kb} and techniques for graph embedding and link prediction.
In Section~\ref{sec:preliminaries}, we introduce preliminary concepts in the field which will be the foundation for understanding the \ac{hake} technique in Section~\ref{sec:method}.
Using the technique, we conduct experiments and present their results in Section~\ref{sec:experiments}.
In Section~\ref{sec:discussion}, we discus the results and highlight some key ideas that can point in directions of future work.
We end with a conclusion in Section~\ref{sec:conclusion}.
