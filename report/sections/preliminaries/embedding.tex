\subsection{Graph Embedding}\label{sec:embedding}

To make graphs more efficient to work with computationally, it can be beneficial to embed them into a vector space\cite{Cai2018}.
This technique is called graph embedding.
The goal of graph embedding is to keep the dimensionality of the target vector space, called the embedding space, low while preserving properties of the graph.
Depending on the properties of interest, the embedding can happen on different levels.
For instance, in weighted graphs, each edge $e$ between two nodes $u$ and $v$ is associated with a weight $w(e)$ that can be used to represent, for instance, the importance or capacity of $e$.
This weight can be represented in the embedding by letting the distance between $u$ and $v$ in the embedding space approximate $w(e)$.
In the context of \acp{kg}, however, the properties of interest lie in both the nodes and the edges.

Generally, embedding a graph can be seen as a problem of learning a function $g$ that maps nodes and edges to vectors in an $\mathbb{R}^d$ embedding space.
When embedding \acp{kg}, this is typically done by defining a score function $f(h, r, t)$ that describes how likely it is for the triple $(h, r, t)$ to exist\cite{Sun2019}.
This function should be constructed in a way that gives high scores to true triples $(h, r, t) \in \mathcal{K}$ and low scores to false triples $(h', r', t') \not\in \mathcal{K}$.
If the score function has this property, learning the embedding can be seen as an optimisation problem with respect to $f$.
Specifically, we wish to maximise the value of $f$ for every $(h, r, t) \in \mathcal{K}$.
We use bold letters to denote embeddings of nodes and edges, for instance $\bm{e} = g(e)$ where $\bm{e}$ is the embedded version of the entity $e$.

Some \ac{kg} embedding techniques are referred to as translational models where, given a triple $(h, r, t)$, the relation $\bm{r}$ acts as a translation from $\bm{h}$ to $\bm{t}$ in the embedding space.
One of the simplest translational models is TransE\cite{Bordes2013}.
The aim of this technique is to satisfy $\bm{h} + \bm{r} = \bm{t}$ for each triple $(h, r, t) \in \mathcal{K}$.
The intuition behind this translation technique is shown in \figurename~\ref{fig:transe} where the triple $(h, r, t)$ is embedded as $\bm{h} = (0.5, 0.5)$, $\bm{t} = (2.5, 1.5)$, and $\bm{r} = (2, 1)$.
This clearly satisfies the objective since $(0.5, 0.5) + (2, 1) = (2.5, 1.5)$.

\begin{figure}[ht]
  \centering\small
  \begin{tikzpicture}
  \draw [lightgray, very thin, step = 0.5] (0, 0) grid (2.9, 1.9);
  \draw [arrow, <->, thin, shorten > = 0pt, shorten < = 0pt] (0, 2) -- (0, 0) -- (3, 0);
  \node [circle, fill, inner sep = 1pt, label = below:$O$] at (0, 0) {};
  \node (h) [circle, fill, inner sep = 1pt, label = $\bm{h}$] at (0.5, 0.5) {};
  \node (t) [circle, fill, inner sep = 1pt, label = $\bm{t}$] at (2.5, 1.5) {};
  \draw [arrow] (h) -- node [midway, auto, fill = white, inner sep = 1pt] {$\bm{h} + \bm{r}$} (t);
\end{tikzpicture}

  \caption{%
    The TransE model\cite{Bordes2013} aims to satisfy $\bm{h} + \bm{r} = \bm{t}$ for every triple $(h, r, t)$.
    In this case $\bm{h} = (0.5, 0.5)$, $\bm{t} = (2.5, 1.5)$, and $\bm{r} = (2, 1)$.%
  }\label{fig:transe}
\end{figure}

It should be noted that the embedding in \figurename~\ref{fig:transe} inhibits the $\mathbb{R}^2$ space since it makes for an intuitive illustration.
However, a graph embedding usually occupies a space of higher dimensions since it allows for more details to be captured.

It is clear that if a TransE embedding satisfies $\bm{h} + \bm{r} = \bm{t}$, we can use $\bm{h} + \bm{r} - \bm{t}$ as a measure of correctness; the closer the value of $f$ is to $0$, the more correct the embedding is.
This leads to the TransE score function which is defined as $f(h, r, t) = - \Vert \bm{h} + \bm{r} - \bm{t} \Vert_p^2$.
Here, $\Vert\cdot\Vert_p$ denotes the $p$-norm which has the general form
\[\Vert \bm{e} \Vert_p = \sqrt[p]{\bm{e}_1^p + \bm{e}_2^p + \cdots + \bm{e}_i^p}.\]
If the TransE embedding is perfect, true triples will have a score of $0$ while false triples will have negative scores.

TransE is a simple model but it lacks the ability to model complicated relations often found in \acp{kg}.
Examples of these are
\begin{enumerate*}
  \item symmetric relations where both $(h, r, t)$ and $(t, r, h)$ are present,
  \item inverse relations where $r$ is an inverse of $r'$ if both $(h, r, t)$ and $(t, r', h)$ exist, and
  \item composed relations where $r$ is a composition of $r'$ and $r''$ if $(h, r, t)$, $(h, r', t')$, and $(t', r'', t)$ are present.
\end{enumerate*}
This is an issue addressed by the RotatE approach\cite{Sun2019}.
This technique embeds triples into a complex embedding space where relations represent rotational translations between entities.
More specifically, given a triple $(h, r, t)$, RotatE aims to satisfy $\bm{h} \circ \bm{r} = \bm{t}$.
Here, $\circ$ denotes the element-wise product of two vectors, meaning that if $\bm{t} = \bm{h} \circ \bm{r}$ then ${\bm{t}}_i = {\bm{h}}_i \bm{r}_i$.
The advantage of RotatE is that it is capable of modelling the different types of relations described above.

In addition to complicated relations, \acp{kg} often also exhibit semantic hierarchies.
For instance, in the medical domain, drugs are grouped in different categories such as antipyretics and vaccines.
The \ac{hake} technique\cite{Zhang2019}, which build upon RotatE, is developed specifically to handle such structures.
The purpose of the technique is to persist semantic hierarchies in the graph embedding.
This is done by dividing the embedding of entities into two parts: one that represents their hierarchical levels and one that distinguishes them on the same level.
Then the relation embeddings act as translations between the entities.

A \ac{kg} embedding can be used for link prediction by passing candidate triples through the score function.
The triples that get assigned the highest scores are most likely to exist.
The \ac{hake} technique currently produces state-of-the-art results for this link prediction task, which is why we take a closer look at it in the next section.
