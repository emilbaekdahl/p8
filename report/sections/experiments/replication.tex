\subsection{Replication}\label{sec:replication}

This experiment is carried out by running the exact experiments presented in\cite{Zhang2019}.
Here, the \ac{hake} technique is evaluated on the three standard datasets WN18RR, FB15k-237, and YAGO3\--10 that are based on the \acp{kg} WordNet, Freebase, and \ac{yago} respectively.
In the original experiments, the authors perform hyperparameter optimisation to find the best parameter configurations for each dataset.
We use the same parameter configurations in the replication experiments.

The results of the replication experiments are shown in Table~\ref{tab:replication}.
The bottom row indicates how many percent the replicated results deviate from the original results.
This deviation is calculated by
\[\frac{\text{replication} - \text{original}}{\text{replication}} \cdot \SI{100}{\percent}.\]

\begin{table*}[ht]
  \centering
  \caption{Results for Replication Experiment}\label{tab:replication}
  \subsection{Replication}\label{sec:replication}

This experiment is carried out by running the exact experiments presented in\cite{Zhang2019}.
Here, the \ac{hake} technique is evaluated on the three standard datasets WN18RR, FB15k-237, and YAGO3\--10 that are based on the \acp{kg} WordNet, Freebase, and \ac{yago} respectively.
In the original experiments, the authors perform hyperparameter optimisation to find the best parameter configurations for each dataset.
We use the same parameter configurations in the replication experiments.

The results of the replication experiments are shown in Table~\ref{tab:replication}.
The bottom row indicates how many percent the replicated results deviate from the original results.
This deviation is calculated by
\[\frac{\text{replication} - \text{original}}{\text{replication}} \cdot \SI{100}{\percent}.\]

\begin{table*}[ht]
  \centering
  \caption{Results for Replication Experiment}\label{tab:replication}
  \subsection{Replication}\label{sec:replication}

This experiment is carried out by running the exact experiments presented in\cite{Zhang2019}.
Here, the \ac{hake} technique is evaluated on the three standard datasets WN18RR, FB15k-237, and YAGO3\--10 that are based on the \acp{kg} WordNet, Freebase, and \ac{yago} respectively.
In the original experiments, the authors perform hyperparameter optimisation to find the best parameter configurations for each dataset.
We use the same parameter configurations in the replication experiments.

The results of the replication experiments are shown in Table~\ref{tab:replication}.
The bottom row indicates how many percent the replicated results deviate from the original results.
This deviation is calculated by
\[\frac{\text{replication} - \text{original}}{\text{replication}} \cdot \SI{100}{\percent}.\]

\begin{table*}[ht]
  \centering
  \caption{Results for Replication Experiment}\label{tab:replication}
  \subsection{Replication}\label{sec:replication}

This experiment is carried out by running the exact experiments presented in\cite{Zhang2019}.
Here, the \ac{hake} technique is evaluated on the three standard datasets WN18RR, FB15k-237, and YAGO3\--10 that are based on the \acp{kg} WordNet, Freebase, and \ac{yago} respectively.
In the original experiments, the authors perform hyperparameter optimisation to find the best parameter configurations for each dataset.
We use the same parameter configurations in the replication experiments.

The results of the replication experiments are shown in Table~\ref{tab:replication}.
The bottom row indicates how many percent the replicated results deviate from the original results.
This deviation is calculated by
\[\frac{\text{replication} - \text{original}}{\text{replication}} \cdot \SI{100}{\percent}.\]

\begin{table*}[ht]
  \centering
  \caption{Results for Replication Experiment}\label{tab:replication}
  \input{sections/experiments/tables/replication.tex}
\end{table*}

By looking at the deviation, it is clear that the different datasets give rise to different results.
On the WordNet dataset, the replication experiment ranges from performing \SI{-1.77}{\percent} worse to \SI{0.172}{\percent} better.
We do not consider this a significant deviation and it could just be due to differences in the initialisation of the entity and relation embeddings.
However, replication results on the Freebase and \ac{yago} datasets are much different from the original results.
Worth noticing is that the replication results are consistently worse than the original.
This indicates that the difference in performance is not only due to differences in the initial embeddings as suggested for the WordNet dataset.
It should be noted, however, that due to limitations on computational power, we have not been able to run the experiments more than a few times per dataset.
This is in contrast to the results in\cite{Zhang2019} which are averages of multiple runs.
Nonetheless, we think that a \SIrange{-3.58}{-11.7}{\percent} deviation is significant.

\end{table*}

By looking at the deviation, it is clear that the different datasets give rise to different results.
On the WordNet dataset, the replication experiment ranges from performing \SI{-1.77}{\percent} worse to \SI{0.172}{\percent} better.
We do not consider this a significant deviation and it could just be due to differences in the initialisation of the entity and relation embeddings.
However, replication results on the Freebase and \ac{yago} datasets are much different from the original results.
Worth noticing is that the replication results are consistently worse than the original.
This indicates that the difference in performance is not only due to differences in the initial embeddings as suggested for the WordNet dataset.
It should be noted, however, that due to limitations on computational power, we have not been able to run the experiments more than a few times per dataset.
This is in contrast to the results in\cite{Zhang2019} which are averages of multiple runs.
Nonetheless, we think that a \SIrange{-3.58}{-11.7}{\percent} deviation is significant.

\end{table*}

By looking at the deviation, it is clear that the different datasets give rise to different results.
On the WordNet dataset, the replication experiment ranges from performing \SI{-1.77}{\percent} worse to \SI{0.172}{\percent} better.
We do not consider this a significant deviation and it could just be due to differences in the initialisation of the entity and relation embeddings.
However, replication results on the Freebase and \ac{yago} datasets are much different from the original results.
Worth noticing is that the replication results are consistently worse than the original.
This indicates that the difference in performance is not only due to differences in the initial embeddings as suggested for the WordNet dataset.
It should be noted, however, that due to limitations on computational power, we have not been able to run the experiments more than a few times per dataset.
This is in contrast to the results in\cite{Zhang2019} which are averages of multiple runs.
Nonetheless, we think that a \SIrange{-3.58}{-11.7}{\percent} deviation is significant.

\end{table*}

By looking at the deviation, it is clear that the different datasets give rise to different results.
On the WordNet dataset, the replication experiment ranges from performing \SI{-1.77}{\percent} worse to \SI{0.172}{\percent} better.
We do not consider this a significant deviation and it could just be due to differences in the initialisation of the entity and relation embeddings.
However, replication results on the Freebase and \ac{yago} datasets are much different from the original results.
Worth noticing is that the replication results are consistently worse than the original.
This indicates that the difference in performance is not only due to differences in the initial embeddings as suggested for the WordNet dataset.
It should be noted, however, that due to limitations on computational power, we have not been able to run the experiments more than a few times per dataset.
This is in contrast to the results in\cite{Zhang2019} which are averages of multiple runs.
Nonetheless, we think that a \SIrange{-3.58}{-11.7}{\percent} deviation is significant.
