\subsection{Future Work}

In its nature, a \ac{kg} is of course a graph and thus exhibits many interesting properties that can be taken into consideration for link prediction.
However, graph embedding-based approaches for link prediction such as TransE, RotatE, and \ac{hake} only consider the triples in isolation when learning the embedding, and thus only learn the structure of the graph.
Immediately, it seems that properties and semantics of relations and entities in the \ac{kg} are lost in this process.
To address the link prediction problem effectively, future work should take into consideration not only the structure of the graph, but also the semantics of entities and their relations.
We think that one of the first steps to consider such semantics is to cluster entities into groups that share similar properties.
In\cite{Chen2019} an efficient approach for graph clustering is introduced.
Here, similarity between nodes is measured both by their structure and their attributes.
This clustering step is applied to star-schema heterogeneous graphs which are defined in the following way.

\begin{definition}[Star-schema heterogeneous graph]
  A heterogeneous graph is a 4-tuple $(V, E, \phi, \psi)$ where $V$ is a set of nodes, $E \subseteq V \times V$ is a set of edges, $\phi \colon V \rightarrow \mathcal{T}_V$ maps each $v \in V$ to a node type ${\mathcal{T}_V}_i \in \mathcal{T}_V$, and $\psi \colon E \rightarrow \mathcal{T}_E$ maps each $e \in E$ to an edge type ${\mathcal{T}_E}_i \in \mathcal{T}_E$.
\end{definition}

A heterogeneous graph is said to be on star-schema form if
\begin{enumerate*}
  \item node attributes are represented as nodes themselves, called attribute nodes, and
  \item nodes that share attributes are connected to the same attribute nodes.
\end{enumerate*}
Furthermore, non-attribute nodes are referred to as hub nodes, i.e.\ we have $\mathcal{T}_V = \{\text{hub}, \text{attribute}\}$.
In context of the medical domain, for instance, diseases and drugs can be hub nodes while symptoms, effects, and causes are attribute nodes.
\figurename~\ref{fig:heterogeneous-star-schema} illustrates the difference between a \enquote{regular} heterogeneous graph and a star-schema heterogeneous graph.
The $d_i$ hub nodes represent diseases and are connected to the $c_i$ and $s_i$ attributes nodes representing causes and symptoms.

\begin{figure}[ht]
  \centering\small
  \subfloat[Heterogeneous]{\centering\begin{tikzpicture}[node distance = 0.75cm]
  \node (d1) {$d_1$};
  \node [below left = 0.75cm and 0.125cm of d1] (d1s1) {$s_1$};
  \node [below = of d1] (d1s2) {$s_2$};
  \node [above = of d1] (d1c1) {$c_1$};
  \draw [-] (d1) -- (d1s1);
  \draw [-] (d1) -- (d1s2);
  \draw [-] (d1) -- (d1c1);

  \node [right = 0.25cm of d1] (d2) {$d_2$};
  \node [above = of d2] (d2c2) {$c_2$};
  \node [below = of d2] (d2s1) {$s_1$};
  \draw [-] (d2) -- (d2c2);
  \draw [-] (d2) -- (d2s1);

  \node [right = 0.25cm of d2] (d3) {$d_3$};
  \node [above = of d3] (d3c1) {$c_1$};
  \node [above right = 0.75cm and 0.125cm of d3] (d3c3) {$c_3$};
  \node [below = of d3] (d3s3) {$s_3$};
  \draw [-] (d3) -- (d3c1);
  \draw [-] (d3) -- (d3c3);
  \draw [-] (d3) -- (d3s3);
\end{tikzpicture}
}\qquad
  \subfloat[Star-schema]{\centering\begin{tikzpicture}[node distance = 0.75cm]
  \node (d1) {$d_1$};
  \node [right = 0.25cm of d1] (d2) {$d_2$};
  \node [right = 0.25cm of d2] (d3) {$d_3$};

  \node [below = of d1] (s1) {$s_1$};
  \node [below = of d2] (s2) {$s_2$};
  \node [below = of d3] (s3) {$s_3$};

  \node [above = of d1] (c1) {$c_1$};
  \node [above = of d2] (c2) {$c_2$};
  \node [above = of d3] (c3) {$c_3$};

  \draw [-] (d1) -- (s2);
  \draw [-] (d1) -- (s1);
  \draw [-] (d1) -- (c1);
  \draw [-] (d2) -- (c2);
  \draw [-] (d2) -- (s1);
  \draw [-] (d3) -- (s3);
  \draw [-] (d3) -- (c1);
  \draw [-] (d3) -- (c3);
\end{tikzpicture}
}
  \caption{The normalisation of the attributes in the left heterogeneous graph produces the star-schema graph on the right.}%
  \label{fig:heterogeneous-star-schema}
\end{figure}

It is clear that we can represent a \ac{kg} as a star-schema heterogeneous graph since it is a special case of a graph.
Since the star-schema structure is capable of capturing similarity both in terms of the properties of entities and how they are related, we think that it may help improving link prediction techniques.
Furthermore, most current \ac{kg} link prediction techniques expect the graph to be homogeneous\cite{Cai2018}.
As such, using star-schema heterogeneous graphs for future research can also uncover advantages and disadvantages of using heterogeneous graphs in this problem setting.

Another issue that can be addressed by avoiding graph embeddings for link prediction is flexibility and scalability.
A graph embedding is fixed to the \ac{kg} it has been learned on.
If we add new knowledge to a \ac{kg}, either entities or relations, the embedding must be re-computed.
This can become expensive and likely infeasible when the frequency of changes or the size of the \ac{kg} increases.
In connection with this, the embedding-based techniques also lack transparency when used for link prediction.
If the model predicts a given link, it is not immediately obvious why that link might exist.
Explainability is a not an issue that most \ac{kg} link prediction techniques currently try to address but it may be worth considering in future research since it is an increasingly important topic\cite{Gilpin2018}.
Similarity measures on graphs are more clear for humans since the specific attributes or nodes that explain the similarity can be pointed out and inspected.

Therefore, we propose that future work in the field of link prediction, and \ac{kg} refinement in general, should take similarity into account.
This can be done with star-schema heterogeneous graphs as a foundation since they offer a unified similarity measure that exhibits promising results.
